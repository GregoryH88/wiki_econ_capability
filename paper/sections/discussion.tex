
\section{Discussion}

One of the brightest results here is that in using the method described for predicting world economies by GDP, we not only can predict Wikipedia editor and article rankings, but outperform the original application. Whereas in Caldarelli the achieve a correlation of 0.4, \cite{Caldarelli} that is on the lower end of our results. A theory to explain this behaviour goes back to the original motivation in creating this method, that of trying to find "capabilities". In Economics there are circumstantial factors that influence a countries capabilities, e.g. Geography or Politics. However in Wikipedia all outputs are only due to the editors' true underlying capabilities. There are no commodities and articles have no intrinsic value until they are written. If this explained why our correlations were higher, it might also be true for other online collaboration sites, where users operate in a within a greater meritocracy.

In Wikipedia there has been discussion about the importance of super-users who represent a small fraction of editors but contribute a majority of content. \cite{website:wikinewsreporter}. It is possible now to take $\alpha$ as a measure of importance of superuser contributions. Since different categories we correlate more highly for different ranges of $\alpha$ it is possible to compare the success of super-users in different categories. Moreover we can also find which categories are most closely modelled by low or negative values of $\alpha$ which represents better articles are made by less fit editors . Ways to determine this pheonomena could be two or more power users fighting over a page can leave it worse than not being touched at all. Another is that perhaps less fit editors have a greater tendency to collaborate. Or that less fit, "newbie" editors start editing quite obvious, and there for ubiquitous articles, which have high quality already.  Whatever the reason,  it shows an anti-competitiveness. It means the contributions of less fit authors are important. This is a departure from the economics domain, where the best fits for GDP are only in the positive / positive $\alpha$-$\beta$ quadrant. 
