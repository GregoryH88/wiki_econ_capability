\begin{abstract}
We introduce a new method to jointly rank the quality of articles and the expertise
 of editors in Categories of Wikipedia, based on the bi-partite network information 
 of who has contributed at least once to an article on the one hand, and what are 
 the articles that have edited at least once by a given author on the other hand.
  We show that this ``reflexive" ranking method exhibits high correlations with 
  usual article quality and user expertise metrics, which account for quality on 
  Wikipedia (that we assume to be a grand truth here). In particular, we find that 
  the quality of an article can be captured very well by our method right after a few 
  edits, while the expertise of editors is captured increasingly better over time. 
  Our results suggest that it is easier to predict the quality of an article from the 
  editors who touched it, rather than editor expertise from articles they have 
  edited.
\end{abstract}