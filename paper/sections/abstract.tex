\begin{abstract}
In open collaboration, knowledge is created and iteratively improved by a multitude of editors, who freely choose what should be their contributions. The quality of knowledge artifacts (e.g. article, source code file) is deeply tied to their individual expertise, and to their ability to achieve collaboration. Conversely, the expertise of contributors is a function of artifacts contributed to. Building upon a large stream of literature on the measurement of article quality and contributor expertise, we propose a recursive metric to measure to jointly measure how editor expertise influences the quality of articles, and how contributions to articles influence editor expertise. This {\it bi-partite network random walker} metric reveals the specific structure of cooperation and how the quality of articles is achieved in Wikipedia. We show that while the wisdom of crowds is well pulled in some categories, more editors per article can also create disvalue.
\end{abstract}