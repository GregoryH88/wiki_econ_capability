\section{Related Work}
To the best of our knowledge, our paper is the first attempt to quantify the value of collective contribution environments from the collaboration structure alone. Our model of {bi-partite random walkers} follows in the lineage of bi-partite networks -- networks with two node types -- in the global economy of countries competing for exporting products \cite{hidalgo2007,hidalgo2009}. The proposed recursive {\it method of reflections} helps understand the competitive advantage (i.e., {\it fitness}) of countries from the types of products they sell, and moreover, whether other countries export similar  (i.e., {\it ubiquitous}) products. The key insight is that most competitive countries are found to not only export non-ubiquitous products, for which they can charge higher price, but also ubiquitous ones. The model of reflections has been improved and complemented in more recent work, mainly to fix its robustness \cite{tacchella2012new, cristelli2012competitors, tacchella2013economic, cristelli2013measuring}. Caldarelli et al. \cite{caldarelli2012network} have proposed an alternative method, based on biased stochastic Markov chains, which helps further understand the mutual influence between nodes in bi-partite networks.

Separately to networks of economic competition, collaboration networks have been studied early on in network sciences, in particular for networks of co-authorship in scientific publications \cite{newman2001}, as well as patterns of self-organization in bi-partite networks of actors-movies\cite{ramasco2004self}. Similarly, the analysis of patterns in the Wikipedia bi-partite networks confirmed the existence of overlapping cliques of densely connected articles and editors  \cite{jesus2009}. In the same study, clustering of densely connected cliques into larger modules \cite{guimera2007module} showed that editors clustered by interest with higher coordinated efforts in densely populated clusters \cite{jesus2009}.

Recent results show that the contribution dynamics of successful open source projects, stem from critical cascades of iterative improvements (commits), which in turn lead to super-linear {\it productive bursts} of contributions \cite{sornette2014howmuch}.  The conditions of emergence of productive bursts, include transparency, self-censored clans, emergent technology, problem front-loading, distributed screening, and modularity \cite{vonkrogh2014designing}. 

However, most empirical results in research on open collaboration was unable to uncover the mechanisms of value creation and performance, mainly because of the {\it bottom-up} emerging properties of peer-production \cite{benkler2002}.  The only notable result in this field was found during a series of Matlab contests aimed at collectively solving NP-hard problems. It was found that work shared as a public good helps individuals quickly reuse existing results, and thus, find better algorithms \cite{gulley2010}. Source code submissions by individuals programmers were tested and benchmarked for their capacity to solve the assigned problem quickly, by executing the compiled code on a computer. Unfortunately, this approach is exclusively feasible for machine executable knowledge (i.e. software code) and in highly controlled environments. For Wikipedia, very rough and sometimes misleading metrics, such edit counts or line, are widely used \cite{editcountitis}.  However, recent efforts have been undertaken to develop stronger metrics, and to assess separately the expertise of contributors \cite{geiger2013}, as well as the quality of articles \cite{wang2013tell}. While useful, these results currently do not allow to precisely attribute the origins of value creation and performance in collective knowledge production.
