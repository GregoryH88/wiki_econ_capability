\section{Conclusion}
%From socio-technical and Wikipedia perspectives this collaborativeness measure could be operationalized to aid the problem of finding and retaining new users. New users have historically been driven away because of harsh first encounters with un-collaborative editors. Imagine a program that would suggest to new editors the less antagonistic parts of Wikipedia in which to start editing. In fact placement suggestion programs already exist, such as ``Suggestbot" \ref{https://en.wikipedia.org/wiki/User:Suggestbot} that are used to try and find next articles of interest for an already-active editor. This suggester would be similar except that it targets friendly Wikipedia categories, and requires no user history. A solution would be to create an feed of the most collaborative categories that a new user could browse through when considering making their first edit. This is a reverse approach form current on-boarding practices, where an interest topic is first chosen and then an edit is made in basically a random-chosen environment. We propose to select the environment first.

%Beyond the Wikipedia, corporate private wikis could also use this measurement to classify their employee's behavior. From a business perspective this algorithm could be run over the entire wiki to produce collaborative results. It could show you how collaborative part, or all of your company wiki is. That statistic could be used to show the health of the wiki's collaboration. It could even be used as a proxy metric for how collaboratively the company is working, and how much the best quality-making users are the most active.

%Overall, it is also practical to run this algorithm on Wiki, because it requires only the binary matrix $M_{ea}$ and simple counting on histories and articles that are already kept in a wiki. It suggests some kind of parsimonious, {\it less is more} mechanism, which has direct implications on the overall cost of evaluating contributions in the complex entanglement of contributions. Namely, the use of the bi-partite network random walker model requires only simple and straightforward data mining, compared with, for instance, a model that would primarily rely on the type of words written by editors.

%We have presented a recursive algorithm based on a {\it bi-partite network random walker} model, which jointly ranks Wikipedia editors by their expertise, and articles by their quality, from a simple input matrix recording which editor has modified a given article. Moreover, upon calibration on 12 categories of Wikipedia articles, the input and the control parameters of the model inform directly on how value is created from the complex network of contributions. It appears that some categories of Wikipedia articles fully benefit from the multiplicity of contributors (i.e. ``collaborativeness"), while for other categories, more contributors per article generate dis-value. The origins of these differences between categories could stem from limited coordination capacity between contributors. The organization of value creation in open collaboration might also be intrinsically different from one type of knowledge to another. Finally, we want to stress the generality of the method we have presented. Similarly to open collaboration in Wikipedia, the proposed algorithm can be applied to a variety of situations, such as social coding (e.g. Github), or collaborative rating (e.g. Amazon or Yelp reviews). Applying the bi-partite network random walker model will help further understand to origins of collective value creation and quality, which are the hallmark of open collaboration.
