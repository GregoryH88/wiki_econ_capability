% THIS IS SIGPROC-SP.TEX - VERSION 3.1
% WORKS WITH V3.2SP OF ACM_PROC_ARTICLE-SP.CLS
% APRIL 2009
%
% It is an example file showing how to use the 'acm_proc_article-sp.cls' V3.2SP
% LaTeX2e document class file for Conference Proceedings submissions.
% ----------------------------------------------------------------------------------------------------------------
% This .tex file (and associated .cls V3.2SP) *DOES NOT* produce:
%       1) The Permission Statement
%       2) The Conference (location) Info information
%       3) The Copyright Line with ACM data
%       4) Page numbering
% ---------------------------------------------------------------------------------------------------------------
% It is an example which *does* use the .bib file (from which the .bbl file
% is produced).
% REMEMBER HOWEVER: After having produced the .bbl file,
% and prior to final submission,
% you need to 'insert'  your .bbl file into your source .tex file so as to provide
% ONE 'self-contained' source file.
%
% Questions regarding SIGS should be sent to
% Adrienne Griscti ---> griscti@acm.org
%
% Questions/suggestions regarding the guidelines, .tex and .cls files, etc. to
% Gerald Murray ---> murray@hq.acm.org
%
% For tracking purposes - this is V3.1SP - APRIL 2009

\documentclass{acm_proc_article-sp}

\usepackage{url}

\begin{document}

\title{Title}
%\subtitle{[Extended Abstract]
%\titlenote{A full version of this paper is available as
%\textit{Author's Guide to Preparing ACM SIG Proceedings Using
%\LaTeX$2_\epsilon$\ and BibTeX} at
%\texttt{www.acm.org/eaddress.htm}}}
%
% You need the command \numberofauthors to handle the 'placement
% and alignment' of the authors beneath the title.
%
% For aesthetic reasons, we recommend 'three authors at a time'
% i.e. three 'name/affiliation blocks' be placed beneath the title.
%
% NOTE: You are NOT restricted in how many 'rows' of
% "name/affiliations" may appear. We just ask that you restrict
% the number of 'columns' to three.
%
% Because of the available 'opening page real-estate'
% we ask you to refrain from putting more than six authors
% (two rows with three columns) beneath the article title.
% More than six makes the first-page appear very cluttered indeed.
%
% Use the \alignauthor commands to handle the names
% and affiliations for an 'aesthetic maximum' of six authors.
% Add names, affiliations, addresses for
% the seventh etc. author(s) as the argument for the
% \additionalauthors command.
% These 'additional authors' will be output/set for you
% without further effort on your part as the last section in
% the body of your article BEFORE References or any Appendices.

\numberofauthors{3} %  in this sample file, there are a *total*
% of EIGHT authors. SIX appear on the 'first-page' (for formatting
% reasons) and the remaining two appear in the \additionalauthors section.
%
\author{
% You can go ahead and credit any number of authors here,
% e.g. one 'row of three' or two rows (consisting of one row of three
% and a second row of one, two or three).
%
% The command \alignauthor (no curly braces needed) should
% precede each author name, affiliation/snail-mail address and
% e-mail address. Additionally, tag each line of
% affiliation/address with \affaddr, and tag the
% e-mail address with \email.
%
% 1st. author
\alignauthor
Maximilian Klein\\%\titlenote{Dr.~Trovato insisted his name be first.}\\
       %\affaddr{School of Information}\\
       %\affaddr{ University of California, Berkeley, 102 South Hall}\\
       %\affaddr{Berkeley, CA 94720}\\
       \email{foo@foo}
% 2st. author
\alignauthor
Thomas Maillart\\%\titlenote{Dr.~Trovato insisted his name be first.}\\
       \affaddr{School of Information}\\
       \affaddr{ University of California, Berkeley, 102 South Hall}\\
       \affaddr{Berkeley, CA 94720}\\
       \email{thomas.maillart@ischool.berkeley.edu}
% 3rd. author
\alignauthor
John Chuang\\%\titlenote{Dr.~Trovato insisted his name be first.}\\
       \affaddr{School of Information}\\
       \affaddr{ University of California, Berkeley, 102 South Hall}\\
       \affaddr{Berkeley, CA 94720}\\
       \email{chuang@ischool.berkeley.edu}
}





\date{30 July 1999}
% Just remember to make sure that the TOTAL number of authors
% is the number that will appear on the first page PLUS the
% number that will appear in the \additionalauthors section.

\maketitle
\begin{abstract}
Abstract
\end{abstract}

% A category with the (minimum) three required fields
\category{H.4}{Information Systems Applications}{Miscellaneous}
%A category including the fourth, optional field follows...
\category{D.2.8}{Software Engineering}{Metrics}[complexity measures, performance measures]

\terms{to be completed}

\keywords{to be completed, if necessary} % NOT required for Proceedings

\section{Introduction}

\section{Method}

1. The current investigation involved collecting historical data of edition and quality metrics, from 10 categories of articles in Wikipedia, with focus on fine-grained edits by contributors to articles.

2. The chosen categories contain between 100 and 1000 articles, and between 100 and 500 contributors have edited at least 100'000 times all the articles over their history. (c.f. table \ref{tab:statistics} for summary statistics on the categories). 

3. For each category, we constructed 10 snapshots of equal number of contributions / we constructed snapshots of 10'000 edits. For each snapshot, we constructed the matrix $\mathbf{M}$ of contributors versus edited articles. For each snapshot, the values in $\mathbf{M}$ are defined as the number of edits made by contributors to each article in the category until the snapshot timestamp. The matrix $\mathbf{M}$ constitutes the basic input for implementing the $\mathbf{FQ}$ algorithm. It has a typical triangular structure as shown on Figure \ref{fig:matrix}. We also collected standard metrics of contributor expertise \cite{} and of article quality \cite{your work}, to test the quality of the $\mathbf{FQ}$ algorithm in the context of group collaboration.

4. We then applied the most general implementation of the $\mathbf{FQ}$ algorithm as developed for modeling the economy and competitiveness of countries \cite{}.

5. While the implementation presented here is strictly similar to \ref{}, the interpretation is slightly different in the context of group collaboration. Indeed, while countries competes for selling products, the hypothesis here is that Wikipedia contributors cooperate, at least in a very informal way, for improving the quality of articles.

6. {\it Refer to problems here, if any.}

\begin{table}
\centering
\caption{Summary statistics for each category}
\begin{tabular}{|c|c|l|} \hline
Non-English or Math&Frequency&Comments\\ \hline
\O & 1 in 1,000& For Swedish names\\ \hline
$\pi$ & 1 in 5& Common in math\\ \hline
\$ & 4 in 5 & Used in business\\ \hline
$\Psi^2_1$ & 1 in 40,000& Unexplained usage\\
\hline\end{tabular}
\label{tab:statistics}
\end{table}


\begin{figure}
\centering
%\epsfig{file=fly.eps}
\caption{Matrix $\mathbf{M}$ ordered by decreasing order of edits on both contributors and articles dimensions.}
\label{fig:matrix}
\end{figure}

\section{Conclusions}

%ACKNOWLEDGMENTS are optional
%\section{Acknowledgments}

%
% The following two commands are all you need in the
% initial runs of your .tex file to
% produce the bibliography for the citations in your paper.
\bibliographystyle{abbrv}
\bibliography{sigproc}  % sigproc.bib is the name of the Bibliography in this case
% You must have a proper ".bib" file
%  and remember to run:
% latex bibtex latex latex
% to resolve all references
%
% ACM needs 'a single self-contained file'!
%
%APPENDICES are optional
%\balancecolumns
%\appendix
%Appendix A
%\section{Headings in Appendices}

\end{document}
